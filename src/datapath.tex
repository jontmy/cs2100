\section{Datapath}
\textit{A collection of components that process data, and perform arithmetic, logical, and memory operations.}

The \textbf{instruction execution cycle} has 5 stages:
\begin{enumerate*}
    \keyitem{fetch}{get instruction from memory, address in \code{PC}}
    \keyitem{decode and operand fetch}{determine the operation and get the operands needed}
    \keyitem{execute (ALU)}{perform the computations to get a result}
    \keyitem{execute (memory access)}{read from or write to memory if necessary}
    \keyitem{write-back}{store the result of the operation}
\end{enumerate*}

\subsection{Fetch}
There are 3 steps in the fetch stage:
\begin{enumerate}
    \item Use \code{PC} to fetch the instruction from memory.
    \item Increment \code{PC} by 4 to get the address of the next instruction.
    \item Feed the output instruction to the decode stage.
\end{enumerate}

The \textbf{instruction memory} element is a \textbf{sequential circuit} with an internal
state which stores the instructions.

When supplied an instruction address $m$, it outputs the content at address $m$.

The \textbf{adder} element takes in the 32-bit \code{PC} and the 32-bit constant 4, 
and outputs the 32-bit sum \code{PC + 4}.

The \textbf{clock signal} is a \textit{square wave} with \textit{rising} and \textit{falling} edges,
and a period controlled by the CPU.

\code{PC} is read in the first half of the \textbf{clock period} and updated
at the \textit{next rising clock edge}.

\subsection{Decode}
There are 3 steps in the decode stage:
\begin{enumerate}
    \item Read the \code{opcode} to determine the instruction type and field lengths.
    \item Read data from all necessary registers.
    \item Feed the operation and operands to the ALU stage.
\end{enumerate}

The \textbf{register file} element is a collection of 32 registers which can be read from or written to.

\begin{tblr}{|X|l|X|l|} \hline
    \textbf{input} & \textbf{bits} & \textbf{output} & \textbf{bits} \\ \hline
    read register 1 & 5 & read data 1 & 32 \\
    read register 2 & 5 & read data 2 & 32 \\ \hline[dashed]
    write register & 5 \\
    write data & 32 \\ \hline
\end{tblr}

The \textbf{\code{RegWrite}} control signal determines whether the instruction
should read from or write to the registers:

\begin{tblr}{|l|l|X|} \hline
    & \textbf{\code{RegWrite}} & \textbf{registers per instruction} \\ \hline
    \code{read} & \code{0} & $\le 2$ \\ \hline[dashed]
    \code{write} & \code{1} & $\le 1$ \\ \hline
\end{tblr}

\subsubsection{\code{R}-format Decoding}
The binary content \code{(Inst)} of R-format instructions map to the inputs as follows:
\begin{itemize}
    \item \code{rs $\equiv$ Inst[25:21] $\mapsto$ rr1}
    \item \code{rt $\equiv$ Inst[20:16] $\mapsto$ rr2}
    \item \code{rd $\equiv$ Inst[15:11] $\mapsto$ wr}
\end{itemize}

\subsubsection{\code{I}-format Decoding}
For I-format instructions, 2 problems arise:
\begin{enumerate}
    \item \code{rt}, not part of \code{imm}, needs to be fed to \code{wr}
    \item \code{imm} needs to be fed to the ALU, not \code{rd2}
\end{enumerate}

A \textbf{multiplexer} chooses the correct \code{Inst} slice to feed to \code{wr}
using the control signal \textbf{\code{RegDst}}:

\begin{tblr}{|l|l|X|} \hline
    & \textbf{\code{RegDst}} & \textbf{input to \code{wr}} \\ \hline
    \code{I}-format & \code{0} & \code{rt $\equiv$ Inst[20:16]} \\ \hline[dashed]
    \code{R}-format & \code{1} & \code{rd $\equiv$ Inst[15:11]} \\ \hline
\end{tblr}

\begin{itemize}
    \item \code{rs $\equiv$ Inst[25:21] $\mapsto$ rr1}
    \item \code{rt $\equiv$ Inst[20:16] $\mapsto$ rr2}
    \item \code{rt $\equiv$ Inst[20:16] $\mapsto$ wr}
\end{itemize}

Another multiplexer chooses the correct 32-bit binary data to feed into the ALU:

\begin{tblr}{|l|l|X|} \hline
    & \textbf{\code{ALUSrc}} & \textbf{input to\code{ ALU opr2}} \\ \hline
    {\code{R}-format \\ or \code{beq}} & \code{0} & \code{rr2 $\equiv$ *rt $\equiv$ *Inst[20:16]} \\ \hline[dashed]
    {\code{I}-format \\ not \code{beq}} & \code{1} & \code{imm $\equiv$ Inst[15:0], sign extended to 32-bits} \\ \hline
\end{tblr}

\subsection{Arithmetic Logic Unit (ALU)}
The ALU performs the arithmetic, shifting, logical, memory, and branching operations.

It takes the operation and operands from the decode stage, performs its computation,
and feeds the result to the memory stage.

\begin{tblr}{|X|l|X|l|} \hline
    \textbf{input} & \textbf{bits} & \textbf{output} & \textbf{bits} \\ \hline
    operand 1 & 32 & is zero? & 1 \\
    operand 2 & 32 & result & 32 \\ \hline
\end{tblr}

The \textbf{\code{ALUControl}} signal determines the operation to perform:

\begin{tblr}{|c|c|} \hline
    \textbf{\code{ALUControl}} & \textbf{operation} \\ \hline
    \code{0000} & \code{and} \\
    \code{0001} & \code{or} \\
    \code{0010} & \code{add} \\
    \code{0110} & \code{sub} \\
    \code{0111} & \code{slt} \\
    \code{1100} & \code{nor} \\ \hline
\end{tblr}

\subsubsection{Branching Instructions}
A multiplexer chooses between the next instruction \code{(PC + 4)} and the branch 
target \code{(BT)} using the control signal \textbf{\code{PCSrc}}:

\begin{tblr}{|l|l|X|} \hline
    \textbf{\code{isZero}} & \textbf{\code{PCSrc}} & \textbf{next instruction} \\ \hline
    \code{false $\equiv$ 0} & \code{0} & \code{PC + 4} \\ \hline[dashed]
    \code{true\phantom{ } $\equiv$ 1} & \code{1} & \code{BT $\equiv$ PC + 4 + imm * 4} \\ \hline
\end{tblr}

Because both register contents need to be compared,
the \code{ALUSrc} control value is set to \code{0} for branching instructions.

\code{BT} is computed as follows:
\begin{enumerate}
    \item left shift by 2 bits the sign-extended \code{imm} from the decode stage (multiplying the offset by 4)
    \item add it to \code{PC + 4} from the adder in the fetch stage
\end{enumerate}

\subsection{Memory}
Only \code{lw/lb} and \code{sw/sb} instructions operate in this stage,
the rest remain idle.

The \textbf{data memory} element exists in RAM:

\begin{tblr}{|X|l|X|l|} \hline
    \textbf{input} & \textbf{bits} & \textbf{output} & \textbf{bits} \\ \hline
    address \code{(addr)} & 32 & read data \code{(rdt)} & 32 \\ \hline[dashed]
    write data \code{(wd)} & 32 \\ \hline
\end{tblr}

\code{addr} is the effective memory address computed by the ALU, and \code{wd}
is connected to \code{rd2} from the register file from the decode stage.

The control signals \textbf{\code{MemRead}} and \textbf{\code{MemWrite}} control
the memory operations:

\begin{tblr}{|c|c|X|} \hline
    \textbf{\code{MemRead}} & \textbf{\code{MemWrite}} & \textbf{action} \\ \hline
    \code{0} & \code{1} & write \code{wd} into \code{addr} \\ \hline[dashed]
    \code{1} & \code{0} & read contents of \code{addr} into \code{rdt} \\ \hline[dashed]
    \code{0} & \code{0} & do nothing \\ \hline[dashed]
    \code{1} & \code{1} & undefined, should not occur \\ \hline
\end{tblr}

\subsubsection{Non-memory Instructions}
A multiplexer chooses between the result of the ALU and the read data from memory,
using the control signal \textbf{\code{MemtoReg}}:

\begin{tblr}{|l|l|X|} \hline
    & \textbf{\code{MemToReg}} & \textbf{output} \\ \hline
    non-memory instruction & \code{0} & ALU output \\ \hline[dashed]
    memory instruction & \code{1} & read data \\ \hline
\end{tblr}

This output is fed to the write-back stage.

\subsection{Write-back}
Stores, branches, and jumps remain idle in this stage as there is nothing to be written.

The result of the memory stage is fed into write data \code{(wd)} of the register file.
