\section{Logic Circuits}

\subsection{Logic Gates}

\begin{tblr}{lX}
    \textbf{\AND gate} & \parbox{\linewidth}{\begin{circuit}
        \node [and gate, inputs=nnn, scale=1.6] (and1) {};
        \draw (and1.input 1) -- node[at end,left]{\code{A}} ++(left:4mm);
        \draw (and1.input 3) -- node[at end,left]{\code{B}} ++(left:4mm);
        \draw (and1.output) -- node[at end,right]{\code{A \land B}} ++(right:4mm);
    \end{circuit}} \\
    \textbf{\OR gate} & \parbox{\linewidth}{\begin{circuit}
        \node [or gate, inputs=nnn, scale=1.6] (or1) {};
        \draw (or1.input 1) -- node[at end,left]{\code{A}} ++(left:4mm);
        \draw (or1.input 3) -- node[at end,left]{\code{B}} ++(left:4mm);
        \draw (or1.output) -- node[at end,right]{\code{A \lor B}} ++(right:4mm);
    \end{circuit}} \\
    \textbf{\NOT gate} & \parbox{\linewidth}{\begin{circuit}
        \node [not gate, scale=1.6] (not) {};
        \draw (not) -- node[at end,left]{\code{A}} ++(left:8mm);
        \draw (not) -- node[at end,right]{\code{A'}} ++(right:14mm);
    \end{circuit}}
\end{tblr}

The \textbf{\code{NAND}} and \textbf{\code{NOR}} operations are the negations of the \code{AND}
and \code{OR} operations respectively:

\begin{tblr}{lX}
    \textbf{\code{NAND} gate} & \parbox{\linewidth}{\begin{circuit}
        \node [nand gate, inputs=nnn, scale=1.6] (nand) {};
        \draw (nand.input 1) -- node[at end,left]{\code{A}} ++(left:4mm);
        \draw (nand.input 3) -- node[at end,left]{\code{B}} ++(left:4mm);
        \draw (nand.output) -- node[at end,right]{\code{(A \land B)'}} ++(right:4mm);
    \end{circuit}} \\
    \textbf{\code{NOR} gate} & \parbox{\linewidth}{\begin{circuit}
        \node [nor gate, inputs=nnn, scale=1.6] (nor) {};
        \draw (nor.input 1) -- node[at end,left]{\code{A}} ++(left:4mm);
        \draw (nor.input 3) -- node[at end,left]{\code{B}} ++(left:4mm);
        \draw (nor.output) -- node[at end,right]{\code{(A \lor B)'}} ++(right:4mm);
    \end{circuit}} \\
\end{tblr}

The \textbf{\code{XOR}} operation outputs 1 only if its inputs are different,
and the \textbf{\code{XNOR}} operation is the complement of that:

\begin{tblr}{lX}
    \textbf{\code{XOR} gate} & \parbox{\linewidth}{\begin{circuit}
        \node [xor gate, inputs=nn, scale=1.6] (xor) {};
        \draw (xor.input 1) -- node[at end,left]{\code{A}} ++(left:4mm);
        \draw (xor.input 2) -- node[at end,left]{\code{B}} ++(left:4mm);
        \draw (xor.output) -- node[at end,right]{\code{A $\oplus$ B}} ++(right:4mm);
    \end{circuit}} \\
    \textbf{\code{XNOR} gate} & \parbox{\linewidth}{\begin{circuit}
        \node [xnor gate, inputs=nnn, scale=1.6] (xnor) {};
        \draw (xnor.input 1) -- node[at end,left]{\code{A}} ++(left:4mm);
        \draw (xnor.input 2) -- node[at end,left]{\code{B}} ++(left:4mm);
        \draw (xnor.output) -- node[at end,right]{\code{(A $\oplus$ B)'}} ++(right:4mm);
    \end{circuit}} \\
\end{tblr}

\code{XNOR} can also be represented by $\odot$.

The \textbf{fan-in} is the number of inputs to a gate.

\begin{defn}{drawing logic circuits}
    Every input must be connected in a working circuit, so use \code{0} or \code{1} depending
    on the gate if you don't have a value to connect.

    If \textbf{complemented literals} are not allowed, use a \code{NOT} gate to negate the input.

    Intersecting wires are denoted by a solid circle \tikz\draw[black, fill=black] (0,0) circle (.4ex);.
\end{defn}


\subsection{Universal Gates}
The set \code{\{AND, OR, NOT\}} is a \textbf{complete set of logic} --- they are sufficient
to build any boolean function.

\code{\{NAND\}} and \code{\{NOR\}} are also complete sets of logic.

\begin{tblr}{colspec = {|lll|}, row{1} = {font=\bfseries},} \hline
    & implementation \#1 & implementation \#2 \\ \hline
    \textbf{SOP} & \code{AND-OR} circuit & \code{NAND} circuit \\ \hline[dashed]
    \textbf{POS} & \code{OR-AND} circuit & \code{NOR} circuit \\ \hline
\end{tblr}


\subsection{Programmable Logic Array (PLA)}
A \textbf{programmable logic array} implements sum-of-products (SOP) circuits,
which allows for multiple outputs.

It has 2 stages:
\begin{enumerate}
    \keyitem*{\AND gates}{product terms}
    \keyitem*{\OR gates}{outputs}
\end{enumerate}

\begin{defn*}{drawing PLA diagrams}
    \begin{enumerate*}
        \item Draw a horizontal line for each input.
        \item Draw an \AND gate for each possible input combination that has at least one \code{1} output minterm.
        \item For each input for every \AND gate, add an inverter \tikz\draw[black] (0,0) circle (.4ex); if the input is \code{0}.
        \item Connect each input of every \AND gate to the corresponding input line with \tikz\draw[black, fill=black] (0,0) circle (.4ex); at the intersection.
        \item Draw a vertical line for each \AND gate output.
        \item Draw an \OR gate for each output.
        \item Connect the output line of each \AND gate to the corresponding \OR gate with \tikz\draw[black, fill=black] (0,0) circle (.4ex); at the intersection if that output is \code{1} .
    \end{enumerate*}
\end{defn*}

Simplified PLA diagrams eliminate the need to draw the \AND and \OR gates.

\begin{defn*}{interpreting simplified PLAs}
    \begin{itemize*}
        \keyitem{inputs}{each input \code{A}, \code{B}, \code{C}, ... is paired with its negated input \code{A'}, \code{B'}, \code{C'}, ... in horizontal lines}
        \keyitem{\AND plane}{solid circles in each vertical line represent the inputs to an \AND gate}
        \keyitem{\OR plane}{solid circles in each horizontal line represent the inputs to an \OR gate}
        \keyitem{ouputs}{each of the \OR gates is (connected to) an output}
    \end{itemize*}
\end{defn*}


\subsection{Half Adder}
The \textbf{half adder} adds 2 bits \code{X} and \code{Y} to produce 2 bits \code{C} and \code{S}:
\begin{itemize}
    \item \code{C = X \land Y}
    \item \code{S = X' \land Y \lor X \land Y' = X $\oplus$ Y}
\end{itemize}

\begin{tblr}{
    colspec = {Q[c, 3ex]Q[c, 3ex]Q[c, 3ex]Q[c, 3ex]},
    row{1} = {font=\bfseries},
    row{2} = {font=\bfseries},
    cell{1}{1} = {c=2}{c},
    cell{1}{3} = {c=2}{c}
}
    \toprule
    inputs & & outputs \\
    \cmidrule[lr]{1-2}
    \cmidrule[lr]{3-4}
    X & Y & C & S \\
    \midrule
    0 & 0 & 0 & 0 \\
    0 & 1 & 0 & 1 \\
    1 & 0 & 0 & 1 \\
    1 & 1 & 1 & 0 \\
    \bottomrule
\end{tblr}


\subsection{Gray Code}
\textbf{Gray code} is an unweighted binary sequence where only \textit{one} bit changes
between consecutive values, with no duplicate values.

An $n$-bit Gray code has $2^n$ values, but there are many possible Gray code sequences.

\begin{defn*}{generating the standard $n$-bit Gray code sequence}
    \begin{enumerate}
        \item Start with the $n$-bit binary values for \code{0} and \code{1}.
        \item Mirror the first existing values, effectively doubling the number of values.
        \item On the $i$-th doubling, replace the \code{0} in the $i+1$th place from the right with \code{1} for each of the reflected values.
        \item Repeat until you have $2^n$ values after $n-1$ doublings.
    \end{enumerate}
\end{defn*}
