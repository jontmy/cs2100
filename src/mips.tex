\section{Registers}
Memory access is slow. To avoid frequent memory access, temporary values are stored in the processor
in \textbf{registers}, which are limited in number.

Registers \textit{do not have data types}, unlike program variables.
Instructions always assume that the data stored in a register is of the correct type.

MIPS has \textbf{32 registers}, which are referred to by \textit{number} or \textit{name}:

\begin{tabularx}{\linewidth}{|l|l|X|} \hline
     \textbf{name} & \textbf{\#} & \textbf{usage} \\ \hline
     \code{\$zero} & \code{0} & constant value zero \\ \hdashline
     \code{\$at} & \code{1} & reserved for the assembler \\ \hdashline
     \code{\$v0} - \code{\$v1} & \code{2} - \code{3} & values for results and expression evaluation \\
     \code{\$a0} - \code{\$a3} & \code{4} - \code{7} & arguments \\
     \code{\$t0} - \code{\$t7} & \code{8} - \code{15} & temporaries \\
     \code{\$s0} - \code{\$s7} & \code{16} - \code{23} & program variables \\
     \code{\$t8} - \code{\$t9} & \code{24} - \code{25} & temporaries \\ \hdashline
     \code{\$k0} - \code{\$k1} & \code{26} - \code{27} & reserved for the OS \\ \hdashline
     \code{\$gp} & \code{28} & global pointer \\
     \code{\$sp} & \code{29} & stack pointer \\
     \code{\$fp} & \code{30} & frame pointer \\
     \code{\$ra} & \code{31} & return address \\ \hline
\end{tabularx}

\section{MIPS Assembly Language}
The \textbf{general instruction syntax} is as follows:

\begin{tabularx}{\linewidth}{|l|X|} \hline
    \multicolumn{2}{|c|}{\code{op \$s0, \$s1, \$s2}} \\ \hline
    & \textbf{description} \\ \hline
    \code{op} & operation \\
    \code{\$s0} & destination register \\
    \code{\$s1} & source register 1 \\
    \code{\$s2} & source register 2 \\ \hline
\end{tabularx}

Each instruction executes a \textit{single} command.
Each line of assembly code contains \textit{at most} one instruction.

Almost all MIPS operations are \textit{register-to-register}

The \# hex synbol is used for comments.

\subsection{Arithmetic Instructions}
\begin{tblr}{|Xll|} \hline
    \textbf{instruction} & \textbf{format} & \textbf{opcode/funct} \\ \hline
    \code{add \$rd, \$rs, \$rt} & \code{R} & $\code{0/20}_{\code{hex}}$ \\
    \code{sub \$rd, \$rs, \$rt} & \code{R} & $\code{0/22}_{\code{hex}}$ \\ \hline[dashed]
    \code{addi \$rt, \$rs, imm} & \code{I} & $\code{8}_{\code{hex}}$ \\ \hline[dashed]
    \code{move \$s0, \$s1} & \code{psuedo} & --- \\ \hline
\end{tblr}

\begin{defn}{immediate subtraction}
    There is no \code{subi} operation as it is equivalent to \code{addi} with a negative constant.
\end{defn}

\begin{defn}{assigning variables}
    The \code{move} instruction is a \textbf{psuedo-instruction} which is translated into
    its MIPS equivalent by the compiler:

    \code{add \$s0, \$s1, \$zero}

    To assign a constant \code{imm} to a register:

    \code{addi \$s0, \$zero, imm}
\end{defn}

The immediate values in \textbf{immediate operations} such as \code{addi} range from
\code{$-2^{15}$} to \code{$2^{15} - 1$}, as the 16-bit 2s complement system is used.

\begin{defn}{setting large constants}
    If 32-bit constants are required, use the \code{lui} operation to load the most-significant 16-bits
    first, followed by an \code{ori} operation to set the least-significant 16-bits:

    \code{lui \$t0, 0xAAAA\phantom{abcdefg}\# t0 = 0xAAAA0000} \linebreak
    \code{ori \$t0, \$t0, 0xF0F0\phantom{ab}\# t0 = 0xAAAAF0F0}
\end{defn}

Use the \textbf{temporary registers} \code{\$t0} to \code{\$t9} to store intermediate results
in complex expressions.

\subsection{Logical Instructions}
\begin{tblr}{|Xll|} \hline
    \textbf{instruction} & \textbf{format} & \textbf{opcode/funct} \\ \hline
    \code{sll \$rd, \$rt, shamt} & \code{R} & $\code{0/00}_{\code{hex}}$ \\
    \code{srl \$rd, \$rt, shamt} & \code{R} & $\code{0/02}_{\code{hex}}$ \\ \hline[dashed]
    \code{and \$rd, \$rs, \$rt} & \code{R} & $\code{0/24}_{\code{hex}}$ \\
    \code{or \$rd, \$rs, \$rt} & \code{R} & $\code{0/25}_{\code{hex}}$ \\
    \code{xor \$rd, \$rs, \$rt} & \code{R} & $\code{0/26}_{\code{hex}}$ \\
    \code{nor \$rd, \$rs, \$rt} & \code{R} & $\code{0/27}_{\code{hex}}$ \\ \hline[dashed]
    \code{andi \$rt, \$rs, imm} & \code{I} & $\code{C}_{\code{hex}}$ \\
    \code{ori \$rt, \$rs, imm} & \code{I} & $\code{D}_{\code{hex}}$ \\
    \code{xori \$rt, \$rs, imm} & \code{I} & $\code{E}_{\code{hex}}$ \\ \hline[dashed]
    \code{lui \$rt, imm} & \code{I} & $\code{F}_{\code{hex}}$ \\ \hline
\end{tblr}

Bitwise \code{NOR} sets the result to \code{1} if both bits are \code{0}, and \code{0} otherwise.

Bitwise \code{XOR} sets the result to \code{1} if both bits are different, and \code{0} otherwise.

\begin{defn}{bitwise \code{NOT}}
    There is no \code{not} operation as it is equivalent to either of the following:
    \begin{itemize}
        \item \code{nor \$rd, \$rs, \$zero}
        \item \code{xor \$rd, \$rs, \$rt}, where \code{\$rt} is all \code{1}s
    \end{itemize}
\end{defn}


The absence of \code{nori} keeps the instruction set small.

In bitshift operations (\code{sll} and \code{srl}), the empty positions are filled with zeros, and the
\textbf{shift amount} is limited to \textbf{5 bits}.

\begin{defn}{multiplication and division}
    For multiplication/division by $k$, if $k$ is a power of $2$,
    use \code{sll} and \code{srl} respectively, setting the \code{shamt} to $k$.

    Otherwise, use a loop.
\end{defn}

\subsection{Memory Instructions}
Memory can be thought of as a \textit{single-dimensional array} of memory location, with each
having an \textbf{address}.

Memory addresses allow access to \textit{bytes} of data, or \textbf{words} of data, which are usually
$2^n$ bytes --- the common unit of transfer between the processor and memory.

\textbf{Word alignment} occurs in memory when words begin at a \textit{byte address} which is a
multiple of the word size --- $2^n$ bytes.

In MIPS, each word is 32 bits (4 bytes), and addresses are 32-bits long --- such that $2^{30}$ words are addressable,
each of which differing by 4.

\begin{tblr}{|Xll|} \hline
    \textbf{instruction} & \textbf{format} & \textbf{opcode/funct} \\ \hline
    \code{lw \$rt, offset(\$rs)} & \code{I} & $\code{23}_{\code{hex}}$ \\
    \code{sw \$rt, offset(\$rs)} & \code{I} & $\code{2B}_{\code{hex}}$ \\ \hline[dashed]
    \code{lb \$rt, offset(\$rs)} & \code{I} & $\code{20}_{\code{hex}}$ \\
    \code{sb \$rt, offset(\$rs)} & \code{I} & $\code{28}_{\code{hex}}$ \\ \hline[dashed]
    \code{ulw \$rt, offset(\$rs)} & \code{pseudo} & --- \\
    \code{usw \$rt, offset(\$rs)} & \code{pseudo} & --- \\ \hline
\end{tblr}

For the word-aligned memory instructions \code{lw} and \code{sw},
the resulting address of \code{\$rs + offset} must be a multiple of 4.

The \code{offset} is a 16-bit 2s complement number.

\subsection{Control Flow Instructions}
\begin{tblr}{|Xll|} \hline
    \textbf{instruction} & \textbf{format} & \textbf{opcode/funct} \\ \hline
    \code{beq \$rs, \$rt, label} & \code{I} & $\code{4}_{\code{hex}}$ \\
    \code{bne \$rs, \$rt, label} & \code{I} & $\code{5}_{\code{hex}}$ \\ \hline[dashed]
    \code{j label} & \code{J} & $\code{2}_{\code{hex}}$ \\ \hline[dashed]
    \code{slt \$rd, \$rs, \$rt} & \code{R} & $\code{0/2A}_{\code{hex}}$ \\
    \code{slti \$rt, \$rs, imm} & \code{I} & $\code{A}_{\code{hex}}$ \\ \hline[dashed]
    \code{blt \$rs, \$rt, label} & \code{pseudo} & --- \\
    \code{bgt \$rs, \$rt, label} & \code{pseudo} & --- \\
    \code{ble \$rs, \$rt, label} & \code{pseudo} & --- \\
    \code{bge \$rs, \$rt, label} & \code{pseudo} & --- \\ \hline
\end{tblr}

\code{slt} and \code{slti} sets the result to \code{1} if \code{\$rs < \$rt} or \code{imm},
and \code{0} otherwise.

\code{beq} and \code{bne} are \textbf{conditional jumps} --- they jump to the label if the condition is true.

Labels are written as \code{<label>:} to the left of a statement, but they are \textit{not} instructions.

A \code{j} instruction is equivalent to \code{beq \$s0 \$s0, <label>}.

The \textbf{program counter} (\code{PC}) typically stores the address of the \textit{next} address to be executed,
and has to be modified by the branching and jump instructions.


\section{Instruction Encoding}
Every MIPS instruction is \textbf{32 bits} in 3 possible formats:

\begin{tblr}{|llll|} \hline
    \textbf{format} & {\textbf{source} \\ \textbf{registers}} & {\textbf{destination} \\ \textbf{registers}} & {\textbf{immediate} \\ \textbf{values}} \\ \hline
    \code{R} & 2 & 1 & 0 \\
    \code{I} & 1 & 1 & 1 \\
    \code{J} & 0 & 0 & 1 \\ \hline
\end{tblr}

\subsection{\code{R}-format}
\begin{tblr}{|X|l|l|l|l|l|l|} \hline
    & \code{opcode} & \code{rs} & \code{rt} & \code{rd} & \code{shamt} & \code{funct} \\ \hline
    bits & 6 & 5 & 5 & 5 & 5 & 6 \\ \hline
\end{tblr}

\begin{itemize}
    \item \code{opcode := 0} for all R-format instructions,
    \item \code{funct} determines the instruction,
    \item \code{shamt := 0}, \code{rd := arith(rs, rt)} for non-shift (arithmetic) instructions, and,
    \item \code{rs := 0}, \code{rd := shift(rt, shamt)} for shift instructions.
\end{itemize}

\subsection{\code{I}-format}
\begin{tblr}{|l|l|l|l|X|} \hline
    & \code{opcode} & \code{rs} & \code{rt} & \code{immediate} \\ \hline
    bits & 6 & 5 & 5 & 16 \\ \hline
\end{tblr}

\begin{itemize}
    \item \code{opcode} determines the instruction since there is no \code{funct} field,
    \item \code{rt} determines the \textit{destination} register since there is no \code{rd} field,
    \item \code{immediate} is a \textit{signed} integer in 2s complement except for bitwise operations where it is \textit{unsigned},
    \item \code{rt := op(rs, immediate)} in general for all non-branching instructions, and,
    \item \code{PC := (PC + 4) + (immediate * 4)} when branching, otherwise \code{PC += 4}, which is the next instruction.
\end{itemize}

For branching instructions, \code{immediate} is the offset from the \textit{next} instruction
to the label of the \textit{target} instruction.

\code{PC} is incremented in multiples of 4 due to word-alignment, which also means that
we can now branch $2^{15} \times 4 = 2^{17}$ bytes away from \code{PC}.

\subsection{\code{J}-format}
\begin{tblr}{|l|l|X|} \hline
    & \code{opcode} & \code{target address} \\ \hline
    bits & 6 & 26 \\ \hline
\end{tblr}

The last 2 bits of every instruction are always \code{00} due to word-alignment,
so we leave them out of the target address.

This leaves with an effective range of 28 bits for the target address, and the remaining 4 bits
are derived from the most significant (leftmost) bits of \code{PC + 4}.

The \textbf{destination address} is therefore:

\begin{tblr}{|l|l|X|l|} \hline
    & \code{(PC+4)[0:4]} & \code{target address} & \code{00} \\ \hline
    bits & 4 & 26 & 2 \\ \hline
\end{tblr}

This creates a maximum jump range of 256 MB.


\subsection{Addressing Modes}
Addressing modes are used to calculate the address of an operand.

\begin{enumerate}[itemsep=0.5em]
    \item \textbf{register addressing} --- \code{add, xor, etc.}: \\ operand is a register
    \item \textbf{immediate addresssing} --- \code{addi, andi, etc.}: \\ operand is a constant within the instruction
    \item \textbf{base/displacement addressing} --- \code{lw, sw} : \\ operand is a memory location at the address the sum of a register and a constant in the instruction
    \item \textbf{PC-relative addressing} --- \code{beq, bne}: \\ address is the sum of the \code{PC} and a constant in the instruction
    \item \textbf{psuedo-direct addressing} --- \code{j}: \\ part of the instruction concantenated with part of the \code{PC}
\end{enumerate}
