\part{MIPS}
An \textbf{instruction set architecture} (ISA) is an abstraction on the interface between
hardware and low-level software.

MIPS is one such ISA, which runs on a \textbf{processor}.

Processors are connected to \textbf{memory} (RAM) via a \textbf{bus},
across which code and data is transferred.

\section{Registers}
Memory access is slow. To avoid frequent memory access, temporary values are stored in the processor
in \textbf{registers}, which are limited in number.

Registers \textit{do not have data types}, unlike program variables.
Instructions always assume that the data stored in a register is of the correct type.

MIPS has \textbf{32 registers}, which are referred to by \textit{number} or \textit{name}:

\begin{tabularx}{\linewidth}{|l|l|X|} \hline
     \textbf{name} & \textbf{\#} & \textbf{usage} \\ \hline
     \code{\$zero} & \code{0} & constant value zero \\ \hdashline
     \code{\$at} & \code{1} & reserved for the assembler \\ \hdashline
     \code{\$v0} - \code{\$v1} & \code{2} - \code{3} & values for results and expression evaluation \\
     \code{\$a0} - \code{\$a3} & \code{4} - \code{7} & arguments \\
     \code{\$t0} - \code{\$t7} & \code{8} - \code{15} & temporaries \\
     \code{\$s0} - \code{\$s7} & \code{16} - \code{23} & program variables \\
     \code{\$t8} - \code{\$t9} & \code{24} - \code{25} & temporaries \\ \hdashline
     \code{\$k0} - \code{\$k1} & \code{26} - \code{27} & reserved for the OS \\ \hdashline
     \code{\$gp} & \code{28} & global pointer \\
     \code{\$sp} & \code{29} & stack pointer \\
     \code{\$fp} & \code{30} & frame pointer \\
     \code{\$ra} & \code{31} & return address \\ \hline
\end{tabularx}

\section{MIPS Assembly Language}
The \textbf{general instruction syntax} is as follows:

\begin{tabularx}{\linewidth}{|l|X|} \hline
    \multicolumn{2}{|c|}{\code{op \$s0, \$s1, \$s2}} \\ \hline
    & \textbf{description} \\ \hline
    \code{op} & operation \\
    \code{\$s0} & destination register \\
    \code{\$s1} & source register 1 \\
    \code{\$s2} & source register 2 \\ \hline
\end{tabularx}

Each instruction executes a \textit{single} command.
Each line of assembly code contains \textit{at most} one instruction.

Almost all MIPS operations are \textit{register-to-register}

The \# hex synbol is used for comments.

\subsection{Arithmetic Instructions}
\begin{tabularx}{\linewidth}{|ll|X|} \hline
    \multicolumn{2}{|l|}{\textbf{instruction}} & \textbf{effect} \\ \hline
    \code{add} & \code{\$s0, \$s1, \$s2} & \code{\$s0 = \$s1 + \$s2} \\
    \code{sub} & \code{\$s0, \$s1, \$s2} & \code{\$s0 = \$s1 - \$s2} \\ \hdashline
    \code{addi} & \code{\$s0, \$s0, <k>} & \code{\$s0 = \$s0 + <k>} \\ \hdashline
    \code{move} & \code{\$s0, \$s1} & \textbf{pseudo-instruction} for \code{\$s0 = \$s1}, equivalent to \code{add \$s0, \$s1, \$zero} \\ \hline
\end{tabularx}

The constants \code{<k>} in \textbf{immediate operations} such as \code{addi} range from
\code{$-2^{15}$} to \code{$2^{15} - 1$}, as the 16-bit 2s complement system is used.

However, if 32-bit constants are required, use the \code{lui} operation to load the most-significant (leftmost) 16-bits
first, followed by an \code{ori} operation to set the least-significant 16-bits.

There is no \code{subi} operation as its equivalent to \code{addi} with a negative constant.

Use the \textbf{temporary registers} \code{\$t0} to \code{\$t9} to store intermediate results
in complex expressions.

\subsection{Logical Instructions}
\begin{tabularx}{\linewidth}{|ll|X|} \hline
    \multicolumn{2}{|l|}{\textbf{instruction}} & \textbf{effect} \\ \hline
    \code{sll} & \code{\$t2, \$s0, <k>} & \code{\$t2 = \$s0 << <k>}, equivalent to \code{t2 *= $2^{\code{<k>}}$} \\
    \code{srl} & \code{\$t2, \$s0, <k>} & \code{\$t2 = \$s0 >> <k>}, equivalent to \code{t2 /= $2^{\code{<k>}}$} \\ \hdashline
    \code{and} & \code{\$t0, \$t1, \$t2} & bitwise \code{AND}, where \code{\$t2} is the \textbf{bit-mask} \\ 
    \code{or} & \code{\$t0, \$t1, \$t2} & bitwise \code{OR}, to force certain bits to \code{1} \\ 
    \code{nor} & \code{\$t0, \$t1, \$t2} & bitwise \code{NOR}, only \code{1} if neither bits are \code{0} \\ 
    \code{xor} & \code{\$t0, \$t1, \$t2} & bitwise \code{XOR}, only \code{1} if both bits are different \\  \hdashline
    \code{andi} & \code{\$t0, \$t1, <k>} & bitwise \code{AND} with a constant \code{k} \\ 
    \code{ori} & \code{\$t0, \$t1, <k>} & bitwise \code{OR} with a constant \code{k} \\
    \code{xori} & \code{\$t0, \$t1, <k>} & bitwise \code{XOR} with a constant \code{k} \\ \hline
\end{tabularx}

In bitshift operations (\code{sll} and \code{srl}), the empty positions are filled with zeros, and the
\textbf{shift amount} is limited to \textbf{5 bits}.

There is no \code{not} operation as its equivalent to \code{nor <dest> <src> \$zero}.

There is no \code{nori} operation as it is rarely used, and not adding it keeps the processor design simple.

\subsection{Memory Instructions}
Memory can be thought of as a \textit{single-dimensional array} of memory location, with each
having an \textbf{address}.

Memory addresses allow access to \textit{bytes} of data, or \textbf{words} of data, which are usually
$2^n$ bytes --- the common unit of transfer between the processor and memory.

\textbf{Word alignment} occurs in memory when words begin at a \textit{byte address} which is a
multiple of the word size --- $2^n$ bytes.

In MIPS, each word is 32 bits (4 bytes), and addresses are 32-bits long --- such that $2^{30}$ words are addressable,
each of which differing by 4.

\begin{tabularx}{\linewidth}{|ll|X|} \hline
    \multicolumn{2}{|l|}{\textbf{instruction}} & \textbf{effect} \\ \hline
    \code{lw} & \code{\$dst, k(\$src)} & loads word at \code{Mem[*src + k]} into register \code{\$dst} \\
    \code{sw} & \code{\$src, k(\$dst)} & stores word in register \code{\$src} into \code{Mem[*dst + k]} \\ \hdashline
    \code{lb} & \code{\$dst, k(\$src)} & loads byte at \code{Mem[*src + k]} into register \code{\$dst} \\
    \code{sb} & \code{\$src, k(\$dst)} & stores byte in register \code{\$src} into \code{Mem[*dst + k]} \\ \hdashline
    \code{ulw} & \code{\$dst, k(\$src)} & psuedo-instruction for loading unaligned words \\
    \code{usw} & \code{\$src, k(\$dst)} & psuedo-instruction for storing unaligned words \\ \hline
\end{tabularx}

Memory operations the only operations which can access data in memory, but there are others
which are less frequently used that are not listed here.

Unlike in \code{lw} and \code{sw}, the displacement constants \code{k} for \code{lb} and \code{sb}
do not need to be multiples of 4.

\subsection{Control Flow Instructions}
\begin{tabularx}{\linewidth}{|ll|X|} \hline
    \multicolumn{2}{|l|}{\textbf{instruction}} & \textbf{effect} \\ \hline
    \code{beq} & \code{\$r1, \$r2, label} & goes to the labelled statement if \code{*r1 == *r2} \\
    \code{bne} & \code{\$r1, \$r2, label} & goes to the labelled statement if \code{*r1 != *r2} \\ \hdashline
    \code{j} & \code{label} & jumps to the labelled statement \\ \hdashline
    \code{slt} & \code{\$dst, \$s1, \$s2} & \code{*dst = *s1 < *s2 ?  1 : 0} \\
    \code{slti} & \code{\$dst, \$src, k} & \code{*dst = *src < k ? 1 : 0} \\ \hline
\end{tabularx}

Labels are written as \code{<label>:} to the left of a statement.

A \code{j} instruction is equivalent to \code{beq \$s0 \$s0, <label>}.

The \textbf{program counter} (\code{PC}) typically stores the address of the \textit{next} address to be executed,
and has to be modified by the branching and jump instructions.


\section{Instruction Encoding}
Every MIPS instruction is \textbf{32 bits} in 3 possible formats:

\begin{tblr}{|llll|} \hline
    \textbf{format} & {\textbf{source} \\ \textbf{registers}} & {\textbf{destination} \\ \textbf{registers}} & {\textbf{immediate} \\ \textbf{values}} \\ \hline
    R & 2 & 1 & 0 \\
    I & 1 & 1 & 1 \\
    J & 0 & 0 & 1 \\ \hline
\end{tblr}

\subsection{R-format}
\begin{tblr}{|X|l|l|l|l|l|l|} \hline
    & \code{opcode} & \code{rs} & \code{rt} & \code{rd} & \code{shamt} & \code{funct} \\ \hline
    bits & 6 & 5 & 5 & 5 & 5 & 6 \\ \hline
\end{tblr}

\begin{itemize}
    \item \code{opcode := 0} for all R-format instructions,
    \item \code{funct} determines the instruction,
    \item \code{shamt := 0}, \code{rd := arith(rs, rt)} for non-shift (arithmetic) instructions, and,
    \item \code{rs := 0}, \code{rd := shift(rt, shamt)} for shift instructions.
\end{itemize}

\subsection{I-format}
\begin{tblr}{|l|l|l|l|X|} \hline
    & \code{opcode} & \code{rs} & \code{rt} & \code{immediate} \\ \hline
    bits & 6 & 5 & 5 & 16 \\ \hline
\end{tblr}

\begin{itemize}
    \item \code{opcode} determines the instruction since there is no \code{funct} field,
    \item \code{rt} determines the \textit{destination} register since there is no \code{rd} field,
    \item \code{immediate} is a \textit{signed} integer in 2s complement except for bitwise operations where it is \textit{unsigned},
    \item \code{rt := op(rs, immediate)} in general for all non-branching instructions, and,
    \item \code{PC := (PC + 4) + (immediate * 4)} when branching, otherwise \code{PC += 4}, which is the next instruction.
\end{itemize}

For branching instructions, \code{immediate} is the offset from the \textit{next} instruction
to the label of the \textit{target} instruction.

\code{PC} is incremented in multiples of 4 due to word-alignment, which also means that
we can now branch $2^{15} \times 4 = 2^{17}$ bytes away from \code{PC}.

\subsection{J-format}
\begin{tblr}{|l|l|X|} \hline
    & \code{opcode} & \code{target address} \\ \hline
    bits & 6 & 26 \\ \hline
\end{tblr}

The last 2 bits of every instruction are always \code{00} due to word-alignment,
so we leave them out of the target address.

This leaves with an effective range of 28 bits for the target address, and the remaining 4 bits
are derived from the most significant (leftmost) bits of \code{PC + 4}.

The \textbf{destination address} is therefore:

\begin{tblr}{|l|l|X|l|} \hline
    & \code{(PC+4)[0:4]} & \code{target address} & \code{00} \\ \hline
    bits & 4 & 26 & 2 \\ \hline
\end{tblr}

This creates a maximum jump range of 256 MB.


\subsection{Addressing Modes}
Addressing modes are used to calculate the address of an operand.

\begin{enumerate}[itemsep=0.5em]
    \item \textbf{register addressing} --- \code{add, xor, etc.}: \\ operand is a register
    \item \textbf{immediate addresssing} --- \code{addi, andi, etc.}: \\ operand is a constant within the instruction
    \item \textbf{base/displacement addressing} --- \code{lw, sw} : \\ operand is a memory location at the address the sum of a register and a constant in the instruction
    \item \textbf{PC-relative addressing} --- \code{beq, bne}: \\ address is the sum of the \code{PC} and a constant in the instruction
    \item \textbf{psuedo-direct addressing} --- \code{j}: \\ part of the instruction concantenated with part of the \code{PC}
\end{enumerate}


\section{Aside: Instruction Set Architecture}
ISAs have two major design philosophies:

\begin{enumerate}[itemsep=0.5em]
    \item \textbf{CISC} --- Complex Instruction Set Computer:\vspace{0.3em}
    \begin{itemize}
        \item single instructions for complex operations
        \item smaller program sizes
        \item complex implementation, little room for hardware optimization
    \end{itemize}
    \item \textbf{RISC} --- Reduced Instruction Set Computer:\vspace{0.3em}
    \begin{itemize}
        \item smaller and simpler instruction set
        \item software combines simpler operations to implement high-level language statements
        \item room for compiler optimization
    \end{itemize}
\end{enumerate}

\subsection{Data Storage Architecture}
There are several common designs:

\begin{enumerate}[itemsep=0.5em]
    \item \textbf{stack architecture}: \\ operands are implcitly popped from the stack
    \item \textbf{accumulator architecture}: \\ one operand is implcitly stored in an accumulator
    \item \textbf{general-purpose register architecture}: \\ operands are stored explicitly in registers, operations are register-memory or register-register
    \item \textbf{memory-memory architecture}: \\ operands are read from memory
\end{enumerate}

\subsection{Memory Architecture}
Memory transfer takes place across buses:

\begin{enumerate}
    \item \textbf{address bus}: $k$-bits, uni-directional from processor to memory
    \item \textbf{data bus}: $n$-bits, bi-directional
    \item \textbf{control lines}: bi-directional, e.g. read/write controls
\end{enumerate}

The address bus feeds addresses from the \textbf{memory address register} to the memory.

Data is written to or read from the \textbf{memory data register}, depending on the R/W control line.

\textbf{Endianness} is the relative ordering of the \textit{bytes} in a word (\textit{not} the bits in a byte!) in memory:

\begin{itemize}
    \item \textbf{big-endian}: MSB stored in smallest address
    \item \textbf{little-endian}: LSB stored in smallest address
\end{itemize}


\section{Datapath}
\textit{A collection of components that process data, and perform arithmetic, logical, and memory operations.}

The \textbf{instruction execution cycle} has 5 stages:
\begin{enumerate*}
    \keyitem{fetch}{get instruction from memory, address in \code{PC}}
    \keyitem{decode and operand fetch}{determine the operation and get the operands needed}
    \keyitem{execute (ALU)}{perform the computations to get a result}
    \keyitem{execute (memory access)}{read from or write to memory if necessary}
    \keyitem{write-back}{store the result of the operation}
\end{enumerate*}

\subsection{Fetch}
There are 3 steps in the fetch stage:
\begin{enumerate}
    \item Use \code{PC} to fetch the instruction from memory.
    \item Increment \code{PC} by 4 to get the address of the next instruction.
    \item Feed the output instruction to the decode stage.
\end{enumerate}

The \textbf{instruction memory} element is a \textbf{sequential circuit} with an internal
state which stores the instructions.

When supplied an instruction address $m$, it outputs the content at address $m$.

The \textbf{adder} element takes in the 32-bit \code{PC} and the 32-bit constant 4, 
and outputs the 32-bit sum \code{PC + 4}.

The \textbf{clock signal} is a \textit{square wave} with \textit{rising} and \textit{falling} edges,
and a period controlled by the CPU.

\code{PC} is read in the first half of the \textbf{clock period} and updated
at the \textit{next rising clock edge}.

\subsection{Decode}
There are 3 steps in the decode stage:
\begin{enumerate}
    \item Read the \code{opcode} to determine the instruction type and field lengths.
    \item Read data from all necessary registers.
    \item Feed the operation and operands to the ALU stage.
\end{enumerate}

The \textbf{register file} element is a collection of 32 registers which can be read from or written to.

\begin{tblr}{|X|l|X|l|} \hline
    \textbf{input} & \textbf{bits} & \textbf{output} & \textbf{bits} \\ \hline
    read register 1 & 5 & read data 1 & 32 \\
    read register 2 & 5 & read data 2 & 32 \\ \hline[dashed]
    write register & 5 \\
    write data & 32 \\ \hline
\end{tblr}

The \textbf{\code{RegWrite}} control signal determines whether the instruction
should read from or write to the registers:

\begin{tblr}{|l|l|X|} \hline
    & \textbf{\code{RegWrite}} & \textbf{registers per instruction} \\ \hline
    \code{read} & \code{0} & $\le 2$ \\ \hline[dashed]
    \code{write} & \code{1} & $\le 1$ \\ \hline
\end{tblr}

\subsubsection{\code{R}-format Decoding}
The binary content \code{(Inst)} of R-format instructions map to the inputs as follows:
\begin{itemize}
    \item \code{rs $\equiv$ Inst[25:21] $\mapsto$ rr1}
    \item \code{rt $\equiv$ Inst[20:16] $\mapsto$ rr2}
    \item \code{rd $\equiv$ Inst[15:11] $\mapsto$ wr}
\end{itemize}

\subsubsection{\code{I}-format Decoding}
For I-format instructions, 2 problems arise:
\begin{enumerate}
    \item \code{rt}, not part of \code{imm}, needs to be fed to \code{wr}
    \item \code{imm} needs to be fed to the ALU, not \code{rd2}
\end{enumerate}

A \textbf{multiplexer} chooses the correct \code{Inst} slice to feed to \code{wr}
using the control signal \textbf{\code{RegDst}}:

\begin{tblr}{|l|l|X|} \hline
    & \textbf{\code{RegDst}} & \textbf{input to \code{wr}} \\ \hline
    \code{I}-format & \code{0} & \code{rt $\equiv$ Inst[20:16]} \\ \hline[dashed]
    \code{R}-format & \code{1} & \code{rd $\equiv$ Inst[15:11]} \\ \hline
\end{tblr}

\begin{itemize}
    \item \code{rs $\equiv$ Inst[25:21] $\mapsto$ rr1}
    \item \code{rt $\equiv$ Inst[20:16] $\mapsto$ rr2}
    \item \code{rt $\equiv$ Inst[20:16] $\mapsto$ wr}
\end{itemize}

Another multiplexer chooses the correct 32-bit binary data to feed into the ALU:

\begin{tblr}{|l|l|X|} \hline
    & \textbf{\code{ALUSrc}} & \textbf{input to\code{ ALU opr2}} \\ \hline
    {\code{R}-format \\ or \code{beq}} & \code{0} & \code{rr2 $\equiv$ *rt $\equiv$ *Inst[20:16]} \\ \hline[dashed]
    {\code{I}-format \\ not \code{beq}} & \code{1} & \code{imm $\equiv$ Inst[15:0], sign extended to 32-bits} \\ \hline
\end{tblr}

\subsection{Arithmetic Logic Unit (ALU)}
The ALU performs the arithmetic, shifting, logical, memory, and branching operations.

It takes the operation and operands from the decode stage, performs its computation,
and feeds the result to the memory stage.

\begin{tblr}{|X|l|X|l|} \hline
    \textbf{input} & \textbf{bits} & \textbf{output} & \textbf{bits} \\ \hline
    operand 1 & 32 & is zero? & 1 \\
    operand 2 & 32 & result & 32 \\ \hline
\end{tblr}

The \textbf{\code{ALUControl}} signal determines the operation to perform:

\begin{tblr}{|c|c|} \hline
    \textbf{\code{ALUControl}} & \textbf{operation} \\ \hline
    \code{0000} & \code{and} \\
    \code{0001} & \code{or} \\
    \code{0010} & \code{add} \\
    \code{0110} & \code{sub} \\
    \code{0111} & \code{slt} \\
    \code{1100} & \code{nor} \\ \hline
\end{tblr}

\subsubsection{Branching Instructions}
A multiplexer chooses between the next instruction \code{(PC + 4)} and the branch 
target \code{(BT)} using the control signal \textbf{\code{PCSrc}}:

\begin{tblr}{|l|l|X|} \hline
    \textbf{\code{isZero}} & \textbf{\code{PCSrc}} & \textbf{next instruction} \\ \hline
    \code{false $\equiv$ 0} & \code{0} & \code{PC + 4} \\ \hline[dashed]
    \code{true\phantom{ } $\equiv$ 1} & \code{1} & \code{BT $\equiv$ PC + 4 + imm * 4} \\ \hline
\end{tblr}

Because both register contents need to be compared,
the \code{ALUSrc} control value is set to \code{0} for branching instructions.

\code{BT} is computed as follows:
\begin{enumerate}
    \item left shift by 2 bits the sign-extended \code{imm} from the decode stage (multiplying the offset by 4)
    \item add it to \code{PC + 4} from the adder in the fetch stage
\end{enumerate}

\subsection{Memory}
Only \code{lw/lb} and \code{sw/sb} instructions operate in this stage,
the rest remain idle.

The \textbf{data memory} element exists in RAM:

\begin{tblr}{|X|l|X|l|} \hline
    \textbf{input} & \textbf{bits} & \textbf{output} & \textbf{bits} \\ \hline
    address \code{(addr)} & 32 & read data \code{(rdt)} & 32 \\ \hline[dashed]
    write data \code{(wd)} & 32 \\ \hline
\end{tblr}

\code{addr} is the effective memory address computed by the ALU, and \code{wd}
is connected to \code{rd2} from the register file from the decode stage.

The control signals \textbf{\code{MemRead}} and \textbf{\code{MemWrite}} control
the memory operations:

\begin{tblr}{|c|c|X|} \hline
    \textbf{\code{MemRead}} & \textbf{\code{MemWrite}} & \textbf{action} \\ \hline
    \code{0} & \code{1} & write \code{wd} into \code{addr} \\ \hline[dashed]
    \code{1} & \code{0} & read contents of \code{addr} into \code{rdt} \\ \hline
    \code{0} & \code{0} & do nothing \\ \hline
    \code{1} & \code{1} & undefined, should not occur \\ \hline
\end{tblr}

\subsubsection{Non-memory Instructions}
A multiplexer chooses between the result of the ALU and the read data from memory,
using the control signal \textbf{\code{MemtoReg}}:

\begin{tblr}{|l|l|X|} \hline
    & \textbf{\code{MemToReg}} & \textbf{output} \\ \hline
    non-memory instruction & \code{0} & ALU output \\ \hline[dashed]
    memory instruction & \code{1} & read data \\ \hline
\end{tblr}

This output is fed to the write-back stage.

\subsection{Write-back}
Stores, branches, and jumps remain idle in this stage as there is nothing to be written.

The result of the memory stage is fed into write data \code{(wd)} of the register file.


\section{Control}
\textbf{Control signals} are generated by a \textbf{control unit} based on the type of instruction.

The control unit is a \textbf{combinational circuit} which takes in the 6-bit opcode
and the 5-bit function code, and outputs the 8 control signals.

\code{funct} is only needed to determine \code{ALUSrc}; every other control signal can be
derived from \code{opcode} alone.

\subsection{\code{ALUControl}}
We start with a simplified 1-bit ALU.

\begin{defn}{1-bit MIPS ALU}
    4 control signals are needed:
    \begin{enumerate}
        \keyitem*{\code{Ainvert}}{\code{1} to invert \code{A}, \code{0} otherwise}
        \keyitem*{\code{Binvert}}{\code{1} to invert \code{B}, \code{0} otherwise}
        \keyitem*{\code{operation} (2 bits)}{select one of the 3 results}
    \end{enumerate}

    5 I/O signals:
    \begin{enumerate}
        \keyitem*{\code{A}}{input}
        \keyitem*{\code{B}}{input}
        \keyitem*{\code{Cin}}{carry in}
        \keyitem*{\code{Cout}}{carry out}
        \keyitem*{\code{result}}{output}
    \end{enumerate}
\end{defn}

These 1-bit ALUs are daisy-chained to form a 32-bit ALU.

The carries from one of the 1-bit ALUs is fed out via \code{Cout} and into the next one via \code{Cin}.

\code{Ainvert} and \code{Binvert} control the multiplexers which select between their inputs and their negated values:

\begin{tblr}{|c|c|} \hline
    \textbf{\code{Xinvert}} & \textbf{output} \\ \hline
    \code{0} & \code{X} \\ \hline[dashed]
    \code{1} & \code{NOT(X)} \\ \hline
\end{tblr}

The \code{operation} control signal selects one of the 3 results from the \code{AND} gate, \code{OR} gate, and the adder \code{ADD}:

\begin{tblr}{|c|c|} \hline
    \textbf{\code{operation}} & \textbf{output} \\ \hline
    \code{00} & \code{AND} \\
    \code{01} & \code{OR} \\ \hline[dashed]
    \code{10} & \code{ADD} \\
    \code{11} & \code{SLT} \\ \hline
\end{tblr}

\begin{defn}{1-bit subtraction}
    \code{Cin} is set to \code{1} due to 2s complement:

    \code{A $-$ B $\equiv$ A $+$ (-B) $\equiv$ A $+$ B' $+$ 1}
\end{defn}

\begin{defn}{\code{ALUControl}}
    In MSB to LSB order, \code{ALUControl} comprises of:
    \begin{enumerate}
        \item \code{Ainvert}
        \item \code{Binvert}
        \item \code{operation}
    \end{enumerate}
\end{defn}

\subsubsection{\code{ALUOp}}
It is necessary to use the \textbf{multi-level decoding approach} to simplify the design process and size of
the main controller.

\textbf{\code{ALUOp}} is an intermediate 2-bit control signal which is generated from \code{opcode}
and used with \code{funct} to determine \code{ALUControl}.

\begin{tblr}{|l|l|X|l|} \hline
    \textbf{instruction} & \textbf{\code{ALUOp}} & \textbf{\code{funct}} & \textbf{\code{ALUControl}} \\ \hline
    \code{lw, sw} & \code{00} & \code{XXXXXX} & \code{0010} \\ \hline[dashed]
    \code{beq} & \code{01} & \code{XXXXXX} & \code{0110} \\ \hline[dashed]
    \code{add} & \code{10} & \code{100000} & \code{0010} \\ \hline[dashed]
    \code{sub} & \code{10} & \code{100010} & \code{0110} \\ \hline[dashed]
    \code{and} & \code{10} & \code{100100} & \code{0000} \\ \hline[dashed]
    \code{or} & \code{10} & \code{100101} & \code{0001} \\ \hline[dashed]
    \code{slt} & \code{10} & \code{101010} & \code{0111} \\ \hline
\end{tblr}

In general, \code{ALUOp} is \code{10} for \code{R}-format instructions.
