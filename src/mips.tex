\part{MIPS}
An \textbf{instruction set architecture} (ISA) is an abstraction on the interface between
hardware and low-level software.

MIPS is one such ISA, which runs on a \textbf{processor}.

Processors are connected to \textbf{memory} (RAM) via a \textbf{bus},
across which code and data is transferred.

\section{Registers}
Memory access is slow. To avoid frequent memory access, temporary values are stored in the processor
in \textbf{registers}, which are limited in number.

Registers \textit{do not have data types}, unlike program variables.
Instructions always assume that the data stored in a register is of the correct type.

MIPS has \textbf{32 registers}, which are referred to by \textit{number} or \textit{name}:

\begin{tabularx}{\linewidth}{|l|l|X|} \hline
     \textbf{name} & \textbf{\#} & \textbf{usage} \\ \hline
     \code{\$zero} & \code{0} & constant value zero \\ \hdashline
     \code{\$at} & \code{1} & reserved for the assembler \\ \hdashline
     \code{\$v0} - \code{\$v1} & \code{2} - \code{3} & values for results and expression evaluation \\
     \code{\$a0} - \code{\$a3} & \code{4} - \code{7} & arguments \\
     \code{\$t0} - \code{\$t7} & \code{8} - \code{15} & temporaries \\
     \code{\$s0} - \code{\$s7} & \code{16} - \code{23} & program variables \\
     \code{\$t8} - \code{\$t9} & \code{24} - \code{25} & temporaries \\ \hdashline
     \code{\$k0} - \code{\$k1} & \code{26} - \code{27} & reserved for the OS \\ \hdashline
     \code{\$gp} & \code{28} & global pointer \\
     \code{\$sp} & \code{29} & stack pointer \\
     \code{\$fp} & \code{30} & frame pointer \\
     \code{\$ra} & \code{31} & return address \\ \hline
\end{tabularx}

\section{MIPS Assembly Language}
The \textbf{general instruction syntax} is as follows:

\begin{tabularx}{\linewidth}{|l|X|} \hline
    \multicolumn{2}{|c|}{\code{op \$s0, \$s1, \$s2}} \\ \hline
    & \textbf{description} \\ \hline
    \code{op} & operation \\
    \code{\$s0} & destination register \\
    \code{\$s1} & source register 1 \\
    \code{\$s2} & source register 2 \\ \hline
\end{tabularx}

Each instruction executes a \textit{single} command.
Each line of assembly code contains \textit{at most} one instruction.

Almost all MIPS operations are \textit{register-to-register}

The \# hex synbol is used for comments.

\subsection{Arithmetic Instructions}
\begin{tabularx}{\linewidth}{|ll|X|} \hline
    \multicolumn{2}{|l|}{\textbf{instruction}} & \textbf{effect} \\ \hline
    \code{add} & \code{\$s0, \$s1, \$s2} & \code{\$s0 = \$s1 + \$s2} \\
    \code{sub} & \code{\$s0, \$s1, \$s2} & \code{\$s0 = \$s1 - \$s2} \\ \hdashline
    \code{addi} & \code{\$s0, \$s0, <k>} & \code{\$s0 = \$s0 + <k>} \\ \hdashline
    \code{move} & \code{\$s0, \$s1} & \textbf{pseudo-instruction} for \code{\$s0 = \$s1}, equivalent to \code{add \$s0, \$s1, \$zero} \\ \hline
\end{tabularx}

The constants \code{<k>} in \textbf{immediate operations} such as \code{addi} range from
\code{$-2^{15}$} to \code{$2^{15} - 1$}, as the 16-bit 2s complement system is used.

However, if 32-bit constants are required, use the \code{lui} operation to load the most-significant (leftmost) 16-bits
first, followed by an \code{ori} operation to set the least-significant 16-bits.

There is no \code{subi} operation as its equivalent to \code{addi} with a negative constant.

Use the \textbf{temporary registers} \code{\$t0} to \code{\$t9} to store intermediate results
in complex expressions.

\subsection{Logical Instructions}
\begin{tabularx}{\linewidth}{|ll|X|} \hline
    \multicolumn{2}{|l|}{\textbf{instruction}} & \textbf{effect} \\ \hline
    \code{sll} & \code{\$t2, \$s0, <k>} & \code{\$t2 = \$s0 << <k>}, equivalent to \code{t2 *= $2^{\code{<k>}}$} \\
    \code{srl} & \code{\$t2, \$s0, <k>} & \code{\$t2 = \$s0 >> <k>}, equivalent to \code{t2 /= $2^{\code{<k>}}$} \\ \hdashline
    \code{and} & \code{\$t0, \$t1, \$t2} & bitwise \code{AND}, where \code{\$t2} is the \textbf{bit-mask} \\ 
    \code{or} & \code{\$t0, \$t1, \$t2} & bitwise \code{OR}, to force certain bits to \code{1} \\ 
    \code{nor} & \code{\$t0, \$t1, \$t2} & bitwise \code{NOR}, only \code{1} if neither bits are \code{0} \\ 
    \code{xor} & \code{\$t0, \$t1, \$t2} & bitwise \code{XOR}, only \code{1} if both bits are different \\  \hdashline
    \code{andi} & \code{\$t0, \$t1, <k>} & bitwise \code{AND} with a constant \code{k} \\ 
    \code{ori} & \code{\$t0, \$t1, <k>} & bitwise \code{OR} with a constant \code{k} \\
    \code{xori} & \code{\$t0, \$t1, <k>} & bitwise \code{XOR} with a constant \code{k} \\ \hline
\end{tabularx}

In bitshift operations (\code{sll} and \code{srl}), the empty positions are filled with zeros, and the
\textbf{shift amount} is limited to \textbf{5 bits}.

There is no \code{not} operation as its equivalent to \code{nor <dest> <src> \$zero}.

There is no \code{nori} operation as it is rarely used, and not adding it keeps the processor design simple.

\subsection{Memory Instructions}
Memory can be thought of as a \textit{single-dimensional array} of memory location, with each
having an \textbf{address}.

Memory addresses allow access to \textit{bytes} of data, or \textbf{words} of data, which are usually
$2^n$ bytes --- the common unit of transfer between the processor and memory.

\textbf{Word alignment} occurs in memory when words begin at a \textit{byte address} which is a
multiple of the word size --- $2^n$ bytes.

In MIPS, each word is 32 bits (4 bytes), and addresses are 32-bits long --- such that $2^{30}$ words are addressable,
each of which differing by 4.

\begin{tabularx}{\linewidth}{|ll|X|} \hline
    \multicolumn{2}{|l|}{\textbf{instruction}} & \textbf{effect} \\ \hline
    \code{lw} & \code{\$dst, k(\$src)} & loads word at \code{Mem[*src + k]} into register \code{\$dst} \\
    \code{sw} & \code{\$src, k(\$dst)} & stores word in register \code{\$src} into \code{Mem[*dst + k]} \\ \hdashline
    \code{lb} & \code{\$dst, k(\$src)} & loads byte at \code{Mem[*src + k]} into register \code{\$dst} \\
    \code{sb} & \code{\$src, k(\$dst)} & stores byte in register \code{\$src} into \code{Mem[*dst + k]} \\ \hdashline
    \code{ulw} & \code{\$dst, k(\$src)} & psuedo-instruction for loading unaligned words \\
    \code{usw} & \code{\$src, k(\$dst)} & psuedo-instruction for storing unaligned words \\ \hline
\end{tabularx}

Memory operations the only operations which can access data in memory, but there are others
which are less frequently used that are not listed here.

Unlike in \code{lw} and \code{sw}, the displacement constants \code{k} for \code{lb} and \code{sb}
do not need to be multiples of 4.

\subsection{Control Flow Instructions}
\begin{tabularx}{\linewidth}{|ll|X|} \hline
    \multicolumn{2}{|l|}{\textbf{instruction}} & \textbf{effect} \\ \hline
    \code{beq} & \code{\$r1, \$r2, label} & goes to the labelled statement if \code{*r1 == *r2} \\
    \code{bne} & \code{\$r1, \$r2, label} & goes to the labelled statement if \code{*r1 != *r2} \\ \hdashline
    \code{j} & \code{label} & jumps to the labelled statement \\ \hdashline
    \code{slt} & \code{\$dst, \$s1, \$s2} & \code{*dst = *s1 < *s2 ?  1 : 0} \\
    \code{slti} & \code{\$dst, \$src, k} & \code{*dst = *src < k ? 1 : 0} \\ \hline
\end{tabularx}

Labels are written as \code{<label>:} to the left of a statement.

A \code{j} instruction is equivalent to \code{beq \$s0 \$s0, <label>}.

The \textbf{program counter} (\code{PC}) typically stores the address of the \textit{next} address to be executed,
and has to be modified by the branching and jump instructions.


\section{Instruction Encoding}
Every MIPS instruction is \textbf{32 bits} in 3 possible formats:

\begin{tblr}{|llll|} \hline
    \textbf{format} & {\textbf{source} \\ \textbf{registers}} & {\textbf{destination} \\ \textbf{registers}} & {\textbf{immediate} \\ \textbf{values}} \\ \hline
    R & 2 & 1 & 0 \\
    J & 1 & 1 & 1 \\
    I & 0 & 0 & 1 \\ \hline
\end{tblr}

\subsection{R-format}
\begin{tblr}{|X|l|l|l|l|l|l|} \hline
    & \code{opcode} & \code{rs} & \code{rt} & \code{rd} & \code{shamt} & \code{funct} \\ \hline
    bits & 6 & 5 & 5 & 5 & 5 & 6 \\ \hline
\end{tblr}

\begin{itemize}
    \item \code{opcode := 0} for all R-format instructions,
    \item \code{funct} determines the instruction,
    \item \code{shamt := 0}, \code{rd := arith(rs, rt)} for non-shift (arithmetic) instructions, and,
    \item \code{rs := 0}, \code{rd := shift(rt, shamt)} for shift instructions.
\end{itemize}

\subsection{I-format}
\begin{tblr}{|l|l|l|l|X|} \hline
    & \code{opcode} & \code{rs} & \code{rt} & \code{immediate} \\ \hline
    bits & 6 & 5 & 5 & 16 \\ \hline
\end{tblr}

\begin{itemize}
    \item \code{opcode} determines the instruction since there is no \code{funct} field,
    \item \code{rt} determines the \textit{destination} register since there is no \code{rd} field,
    \item \code{immediate} is a \textit{signed} integer in 2s complement except for bitwise operations where it is \textit{unsigned},
    \item \code{rt := op(rs, immediate)} in general for all non-branching instructions, and,
    \item \code{PC := (PC + 4) + (immediate * 4)} when branching, otherwise \code{PC += 4}, which is the next instruction.
\end{itemize}

For branching instructions, \code{immediate} is the offset from the \textit{next} instruction
to the label of the \textit{target} instruction.

\code{PC} is incremented in multiples of 4 due to word-alignment, which also means that
we can now branch $2^{15} \times 4 = 2^{17}$ bytes away from \code{PC}.

\subsection{J-format}
\begin{tblr}{|l|l|X|} \hline
    & \code{opcode} & \code{target address} \\ \hline
    bits & 6 & 26 \\ \hline
\end{tblr}

The last 2 bits of every instruction are always \code{00} due to word-alignment,
so we leave them out of the target address.

This leaves with an effective range of 28 bits for the target address, and the remaining 4 bits
are derived from the most significant (leftmost) bits of \code{PC + 4}.

The \textbf{destination address} is therefore:

\begin{tblr}{|l|l|X|l|} \hline
    & \code{(PC+4)[0:4]} & \code{target address} & \code{00} \\ \hline
    bits & 4 & 26 & 2 \\ \hline
\end{tblr}

This creates a maximum jump range of 256 MB.


\subsection{Addressing Modes}
Addressing modes are used to calculate the address of an operand.

\begin{enumerate}[itemsep=0.5em]
    \item \textbf{register addressing} --- \code{add, xor, etc.}: \\ operand is a register
    \item \textbf{immediate addresssing} --- \code{addi, andi, etc.}: \\ operand is a constant within the instruction
    \item \textbf{base/displacement addressing} --- \code{lw, sw} : \\ operand is a memory location at the address the sum of a register and a constant in the instruction
    \item \textbf{PC-relative addressing} --- \code{beq, bne}: \\ address is the sum of the \code{PC} and a constant in the instruction
    \item \textbf{psuedo-direct addressing} --- \code{j}: \\ part of the instruction concantenated with part of the \code{PC}
\end{enumerate}


\section{Aside: Instruction Set Architecture}
ISAs have two major design philosophies:

\begin{enumerate}[itemsep=0.5em]
    \item \textbf{CISC} --- Complex Instruction Set Computer:\vspace{0.3em}
    \begin{itemize}
        \item single instructions for complex operations
        \item smaller program sizes
        \item complex implementation, little room for hardware optimization
    \end{itemize}
    \item \textbf{RISC} --- Reduced Instruction Set Computer:\vspace{0.3em}
    \begin{itemize}
        \item smaller and simpler instruction set
        \item software combines simpler operations to implement high-level language statements
        \item room for compiler optimization
    \end{itemize}
\end{enumerate}

\subsection{Data Storage Architecture}
There are several common designs:

\begin{enumerate}[itemsep=0.5em]
    \item \textbf{stack architecture}: \\ operands are implcitly popped from the stack
    \item \textbf{accumulator architecture}: \\ one operand is implcitly stored in an accumulator
    \item \textbf{general-purpose register architecture}: \\ operands are stored explicitly in registers, operations are register-memory or register-register
    \item \textbf{memory-memory architecture}: \\ operands are read from memory
\end{enumerate}

\subsection{Memory Architecture}
Memory transfer takes place across buses:

\begin{enumerate}
    \item \textbf{address bus}: $k$-bits, uni-directional from processor to memory
    \item \textbf{data bus}: $n$-bits, bi-directional
    \item \textbf{control lines}: bi-directional, e.g. read/write controls
\end{enumerate}

The address bus feeds addresses from the \textbf{memory address register} to the memory.

Data is written to or read from the \textbf{memory data register}, depending on the R/W control line.

\textbf{Endianness} is the relative ordering of the \textit{bytes} in a word (\textit{not} the bits in a byte!) in memory:

\begin{itemize}
    \item \textbf{big-endian}: MSB stored in smallest address
    \item \textbf{little-endian}: LSB stored in smallest address
\end{itemize}