\section{ISA Overview}

An \textbf{instruction set architecture} (ISA) is an abstraction on the interface between
hardware and low-level software, which runs on a \textbf{processor}.

Processors are connected to \textbf{memory} (RAM) via a \textbf{bus},
across which code and data is transferred.

ISAs have two major design philosophies:

\begin{enumerate}[itemsep=0.5em]
    \item \textbf{CISC} --- Complex Instruction Set Computer:\vspace{0.3em}
    \begin{itemize}
        \item single instructions for complex operations
        \item smaller program sizes
        \item complex implementation, little room for hardware optimization
    \end{itemize}
    \item \textbf{RISC} --- Reduced Instruction Set Computer:\vspace{0.3em}
    \begin{itemize}
        \item smaller and simpler instruction set
        \item software combines simpler operations to implement high-level language statements
        \item room for compiler optimization
    \end{itemize}
\end{enumerate}

\subsection{Data Storage Architecture}
\begin{enumerate}[itemsep=0.5em]
    \item \textbf{stack architecture}: \\ operands are implcitly popped from the stack
    \item \textbf{accumulator architecture}: \\ one operand is implcitly stored in an accumulator
    \item \textbf{general-purpose register architecture}: \\ operands are stored explicitly in registers, operations are register-memory or register-register
    \item \textbf{memory-memory architecture}: \\ operands are read from memory
\end{enumerate}

\subsection{Memory Architecture}
Memory transfer takes place across buses:

\begin{enumerate}
    \item \textbf{address bus}: $k$-bits, uni-directional from processor to memory
    \item \textbf{data bus}: $n$-bits, bi-directional
    \item \textbf{control lines}: bi-directional, e.g. read/write controls
\end{enumerate}

The address bus feeds addresses from the \textbf{memory address register} to the memory.

Data is written to or read from the \textbf{memory data register}, depending on the R/W control line.

\textbf{Endianness} is the relative ordering of the \textit{bytes} in a word (\textit{not} the bits in a byte!) in memory:

\begin{itemize}
    \item \textbf{big-endian}: MSB stored in smallest address
    \item \textbf{little-endian}: LSB stored in smallest address
\end{itemize}

\section{Instruction Set Encoding}
Instructions consist of an \textbf{opcode} --- a unique code to identify the operation 
--- as well as \textbf{operands}.

In an ISA with \textbf{fixed-length instructions}, we need to fit multiple sets of instruction
types each with the same number of bits.

We use the \textbf{expanding opcode} scheme in which the opcode has variable length
for different instructions.

\begin{defn}{maximizing the total number of instructions}
    Maximize the number of instructions in each set,
    starting from the \textit{largest} set to the smallest.
\end{defn}

\begin{defn}{minimizing the total number of instructions}
    Maximize the number of instructions in each set,
    starting from the \textit{smallest} set to the largest.
\end{defn}