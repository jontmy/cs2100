\part{Data Representation}
Data is represented as a sequence of \textbf{bits} --- \code{0} or \code{1}.

8 bits form a \textbf{byte}, and a fixed multiple of bytes in a computer architecture
forms a \textbf{word}.

\begin{defn}{number of bits \convert number of values}
    $n$ bits can represent up to $2^n$ values.

    Conversely, we need at least $\lceil \log_2 m \rceil$ bits to represent $m$ values.
\end{defn}

\section{Number Systems}
Number systems are \textbf{weighted-positional} systems, with
the \textbf{decimal} number system having a \textbf{base} (or \textbf{radix}) of 10.

\begin{tblr}{|lll|} \hline
    \textbf{number system} & \textbf{base} & \textbf{prefix} \\ \hline
    binary & \code{2} & \code{0b} \\
    octal & \code{8} & \code{0} \\ \hline[dashed]
    decimal & \code{10} & --- \\ \hline[dashed]
    hexadecimal & \code{16} & \code{0x} \\ \hline
\end{tblr}

The digit to the left of the decimal point has a weight of $10^0$, with the exponent
\textit{incrementing} leftward and \textit{decrementing} rightward.

\begin{defn}{decimal \to binary}
    For \textit{whole numbers}, use \textbf{repeated division-by-two}:
    \begin{enumerate}
        \item divide the value by 2, and record the remainder
        \item prepend the remainder to the binary representation
        \item repeat until the value is 0
    \end{enumerate}

    For \textit{fractions}, use \textbf{repeated multiplication-by-two}:
    \begin{enumerate}
        \item multiply the value by 2, and record the carry
        \item append the carry to the binary representation
        \item repeat until the value is 1
        \item append the 1 to the binary representation
    \end{enumerate}
\end{defn}

\begin{defn}{binary \to decimal}
    Use \textbf{weighted summation}:
    \begin{enumerate}
        \item assign the digit to the left of the binary point a weight of $2^0$
        \item increment the weight exponents leftward 
        \item decrement the weight exponents rightward
        \item multiply each digit by its weight
        \item sum the products
    \end{enumerate}
\end{defn}

These conversions generalize to any base!

Converting between 2 bases can be done with base-10 as an intermediary.

\begin{defn*}{binary \convert base-$2^k$}
    \begin{enumerate}
        \item partition the binary representation into $k$-bit chunks outward from the binary point
        \item convert each chunk to base-10
        \item convert each base-10 value to base-$2^k$
    \end{enumerate}

    Apply the reverse direction for converting from base-$2^k$ to binary.
\end{defn*}

\subsection{Negative Numbers}
\textbf{Signed numbers} include all positive and negative values, unlike \textbf{unsigned numbers} which
only include non-negative values.

We will consider $n$-bit values in this section.

\begin{defn}{sign-and-magnitude}
    The MSB is the \textbf{sign bit} --- \code{0} for positive, \code{1} for negative.

    Negate a value by inverting the sign bit.

    \begin{itemize}
        \keyitem*{range}{$-2^{n-1} + 1$ to $2^{n-1} - 1$}
        \keyitem*{zeros}{$-0_{10}$ and $+0_{10}$}
    \end{itemize}
\end{defn}

\begin{defn}{1s complement}
    An $n$-bit number $x$ has a negated value $-x = 2^n - x - 1$.

    Negate a value by inverting \textit{all the bits}.

    \begin{itemize}
        \keyitem*{range}{$-2^{n-1} + 1$ to $2^{n-1} - 1$}
        \keyitem*{zeros}{$-0_{10}$ (all \code{1s}) and $+0_{10}$ (all \code{0}s)}
    \end{itemize}

    The weight of the MSB is $-2^{n-1} + 1$.
\end{defn}

\begin{defn}{2s complement}
    An $n$-bit number $x$ has a negated value $-x = 2^n - x$.

    Negate a value by inverting all the bits to the left of the \textit{rightmost} \code{1}.

    \begin{itemize}
        \keyitem*{range}{$-2^{n-1}$ to $2^{n-1} - 1$}
        \keyitem*{zeros}{$0_{10}$ (all \code{0s})}
    \end{itemize}

    Unlike sign-and-magnitude and 1s complement, 2s complement does not have duplicate zeros.

    The weight of the MSB is $-2^{n-1}$.

    The MSB also represents the sign  --- \code{0} for positive, \code{1} for negative.
\end{defn}

Negating a fractional number in a complement representation is no different
than negating a whole number!

\begin{defn}{excess representation}
    Distribute positive and negative values by a simple addition or subtraction by $k$,
    which is also known as the \textbf{bias}.
    
    $k$ is typically $2^n - 1$ for an $n$-bit number.

    The binary value in excess-$k$ \textit{subtracts} $k$ from its decimal value.

    Conversely, the decimal value in excess-$k$ \textit{adds} $k$ to its binary value.
\end{defn}

\subsection{Overflow}
\textbf{Overflow} occurs when addition or subtraction goes beyond the fixed range of a signed number.

Subtraction is equivalent to addition of the negated value.

Note that in 1s addition, the carry out of the MSB is \textit{added} to the result,
but this carry out is \textit{discarded} in 2s addition.

\begin{defn}{detecting overflow}
    If the MSB (sign bit) of the result is different from the MSBs of the operands,
    then overflow has occurred.

    Note that overflow can only occur when the operands have the same sign,
    i.e., adding a negative number to a positive number will \textit{never} result
    in overflow.
\end{defn}

\subsection{Real Numbers}
\textbf{Fixed-point representation} fixes the number of bits for the whole number and fractional parts,
which limits the range of values.

The IEEE 754 \textbf{floating-point representation} resolves this by allocating a fixed number of bits to the
\textbf{sign}, \textbf{exponent}, and \textbf{mantissa}.

\begin{tblr}{|llllll|} \hline
    \textbf{precision} & \textbf{bits} & \textbf{sign} & \textbf{exp.} & \textbf{mantissa} & \textbf{bias} \\\hline
    single & 32 & 1 & 8 & 23 & 127 \\ \hline[dashed]
    double & 64 & 1 & 11 & 52 & 1023 \\ \hline
\end{tblr}

The \textbf{mantissa} is \textbf{normalized} with an implicit leading 1 bit, e.g. $110.1_2$ is normalized to
$1.101_2 \times 2^2$, and only $101$ is stored in the mantissa.

\begin{defn*}{decimal \to IEEE 754 floating-point representation}
    \begin{enumerate}
        \item convert the decimal value to normalized binary, i.e., $\pm 1.xxx \times 2^y$
        \item determine the sign bit --- \code{0} for positive, and \code{1} for negative
        \item add the bias to the exponent $y$ --- $+127$ for single precision, and $+1023$ for double precision
        \item convert the exponent to binary
        \item concatenate the sign bit, exponent, and mantissa $xxx$
    \end{enumerate}
\end{defn*}